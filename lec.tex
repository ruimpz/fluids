\documentclass{article}
\usepackage[margin=3cm]{geometry}
\usepackage{microtype}
\usepackage{amsmath}
\usepackage{amssymb}
\usepackage{physics}
\usepackage{tikz}

\begin{document}
% Introduction
\title{Notes on Fluids and Plasmas in Astrophysics}
\date{}
\maketitle


\section{Lecture 1 - 17/09/2019}

A fluid

\begin{itemize}
\item flows (yes... indeed)
\item is typically a liquid or a gas
\item is composed of electrically neutral atoms or molecules. 
\end{itemize}

The fluid model of a system is a macroscopic description of a system as a
manifold in which macroscopic forces act.

The fluid description of a system is a valid one if the mean free path of it's
constituents is much smaller than the characteristic length of the system.

\begin{equation*}
  \lambda_c << L
\end{equation*}

\section{Lecture 2 - 19/09/2019}

Most fluids present some resistance to stress and are therefore viscous. If this
does not happen that fluid is called an \textbf{ideal fluid}.

\subsection{Derivatives}

We start by establishing some notation.


\subsection{Eulerian Derivative}

The Eulerian derivative describe the change of a variable with respect to time
for a fixed point in space. It can be identified as a partial time derivative.

\subsection{Lagrangian Derivative}

The Lagrangian derivative describes the change of a variable with respect to
both physical and temporal displacement. It can be identified with a total derivative.

\begin{equation*}
  \mathrm{d} Q=\frac{\partial Q}{\partial t} \mathrm{d} t+\frac{\partial Q}{\partial x} \mathrm{d} x+\frac{\partial Q}{\partial y} \mathrm{d} y+\frac{\partial Q}{\partial z} \mathrm{d} z=\frac{\partial Q}{\partial t} \mathrm{d} t+(\mathrm{d} \vec{r} \cdot \nabla) Q
\end{equation*}

In particular, if one is describing a change in velocity, one may write

\begin{equation*}
  \frac{\mathrm{d} \vec{v}}{\mathrm{d} t}=\frac{\partial \vec{v}}{\partial t}+(\vec{v} \cdot \nabla) \vec{v}
\end{equation*}

\subsection{Conservation Laws}
\label{sec:conservation_laws}

Let $V$ be an element of fluid limited by a surface $S$. The flux through an
oriented element of surface $\dd{\vec{S}}$ is given by

\begin{equation*}
  \frac{\partial}{\partial t} \int_{V} \rho \mathrm{d} V=-\oint_{S} \rho \vec{v} \cdot \vec{n} \mathrm{d} S
\end{equation*}

By the divergence theorem we have

\begin{equation*}
  \int_{V}\left[\frac{\partial \rho}{\partial t}+\nabla \cdot(\rho \vec{v})\right] \mathrm{d} V=0
\end{equation*}

and follows the continuity equation

\begin{equation}
  \label{eq:continuity}
  \frac{\partial \rho}{\partial t}+\nabla \cdot(\rho \vec{v})=0
\end{equation}


\subsection{Equation of Motion}

Let $\delta V$ be an element of fluid of mass $\rho \delta V$. From Newton's
second law

\begin{eqnarray*}
  \rho \delta V \frac{\partial \vec{v}}{\partial t}=\delta \vec{F}_{v o l u m e}+\delta \vec{F}_{\text {surface}} 
\end{eqnarray*}

Let $\vec{f}$ be a volumetric force density per unit mass (also called
acceleration density). We have the relation $\delta \vec{F}_{\text{volume}} =
\rho \delta V \vec{f}$. It is then natural to define a \textbf{total pressure}
tensor $P_{ij}$ \footnote{We study fluids exclusively in euclidean geometry,
  that means $g_{\mu \nu} = \delta_{\mu \nu}$. To keep the notes as
  comprehensible as possible, we make no distinction between covariant and
  contravariant entities, and all indices will be written as subscripts.} that can be interpreted as the component $i$ of the pressure
exerted in a surface of normal vector component $j$.

\begin{center}
\begin{tikzpicture}
  \draw (-2, 0) arc (-90:90:2);
  \draw[thick,->] (0,2) -- (2,2) node[anchor=north west] {$\vec{n}$};
  \draw[thick,->] (0,2) -- (-1.73,1) node[anchor=north west] {$\vec{f}$};
\end{tikzpicture}
\end{center}

Then we have

\begin{equation*}
  \vec{f}_{\text {surface}}=\int_{S} \mathrm{d} \vec{f}_{\text {surface}}=\oint_{S} P_{i j} \mathrm{d} S_{j}
\end{equation*}

Applying the diverge theorem in the right side

\begin{equation*}
  \oint_{S} P_{i j} \mathrm{d} S_{i}=\int_{V} \frac{\partial P_{i j}}{\partial x_{j}} \mathrm{d} V_{i}
\end{equation*}

As the tensor is defined positively in the outwards direction, we add a negative
sign for convenience. We then get the relation

\begin{equation}
  \label{eq:euler}
  \rho \frac{\mathrm{d} \vec{v}}{\mathrm{d} t}=\rho \vec{f}-\nabla \cdot p
\end{equation}


\section{Lecture 3 - 24/09/2019}

\subsection{Fluid in Static Equilibrium}

For a fluid in static equilibrium ($v = 0$), the force that acts on an element
of are is perpendicular to it's normal vector. That is

\begin{equation}
  \label{eq:static_eq_P}
  P_{ij} = p \delta_{ij}
\end{equation}

We may now give an alternative definition of a fluid, as a substance for which
movement is induced whenever a superficial force with an tangential component
(also called a \textit{shear force}) is exerted.

In this case, from equation \ref{eq:euler} follows

\begin{equation*}
  \dd{F_{\text{sup}}} = -p \dd{S}
\end{equation*}

However, one may not forget that the relation \ref{eq:static_eq_P} is no longer
true for a fluid in motion. One should expect a shear force to carry momentum
from the faster layers of fluid to the slower ones. These shear forces are
described expressed through a \textbf{viscosity coefficient}. In other words,
equation \ref{eq:static_eq_P} is only true for non-viscous (ideal) fluids.

\subsection{Ideal Fluids}

For an ideal fluid the equation of motion reduces to

\begin{equation*}
  \rho \frac{\mathrm{d} \vec{v}}{\mathrm{d} t}=-\nabla p+\rho \vec{f}
\end{equation*}

which may be rewritten as

\begin{equation}
  \label{eq:euler_ideal_fluid}
  \rho \frac{\partial \vec{v}}{\partial t}+\rho(\vec{v} \cdot \nabla) \vec{v}=-\nabla p+\rho \vec{f}
\end{equation}

\subsection{Equation of Energy}

Remembering the \textbf{First Law of Thermodynamics}

\begin{equation}
  \label{eq:3rd_law_thermo}
  \mathrm{d} Q=\mathrm{d} U+p \mathrm{d} V
\end{equation}

we aim to apply it to a fluid in motion.

We start by noticing that a sufficiently small element of fluid is trivially in
thermodynamic equilibrium. Let $\delta m$ be the mass of an element of fluid.
Let also $\dd{q} = \frac{ \dd{Q} }{\delta m}$ and $\dd{\epsilon} =
\frac{\dd{u}}{\delta m}$. Then

\begin{equation*}
  \frac{\mathrm{d} q}{\mathrm{d} t}=\frac{\mathrm{d} \epsilon}{\mathrm{d} t}+p \frac{\mathrm{d}}{\mathrm{d} t}\left(\frac{1}{\rho}\right)
\end{equation*}

From equation \ref{eq:continuity}

\begin{equation*}
  p \frac{\mathrm{d}}{\mathrm{d} t}\left(\frac{1}{\rho}\right)=  -\frac{p}{\rho^{2}} \frac{\mathrm{d} \rho}{\mathrm{d} t} =  \frac{p}{\rho} \nabla \cdot \vec{v} 
\end{equation*}

Substituting

\begin{equation*}
  \rho \frac{\mathrm{d} \epsilon}{\mathrm{d} t}+p \nabla \cdot \vec{v}=\rho \frac{\mathrm{d} q}{\mathrm{d} t}
\end{equation*}

If one defines

\begin{equation}
  \label{eq:L_energy}
  -\mathcal{L}=\rho \frac{\mathrm{d} q}{\mathrm{d} t}
\end{equation}

the relation simplifies to

\begin{equation}
  \label{eq:energy}
  \rho \frac{\mathrm{d} \epsilon}{\mathrm{d} t}+p \nabla \cdot \vec{v}=-\mathcal{L}
\end{equation}

If the variations of heat are a case of simply \textbf{heat flux}

\begin{equation*}
  \mathcal{F}=-k \nabla T
\end{equation*}

with $k$ being the thermal conductivity coefficient. $\mathcal{F}$ is related to
$\mathcal{L}$ through

\begin{equation*}
  \mathcal{L}=\nabla \cdot \mathcal{F}=-\nabla \cdot(k \nabla T)
\end{equation*}

Substituting in equation \ref{eq:energy}

\begin{equation*}
  \rho\left(\frac{\partial \epsilon}{\partial t}+(\vec{v} \cdot \nabla) \epsilon\right)+p \nabla \cdot \vec{v}-\nabla \cdot(k \nabla T)=0
\end{equation*}

Other phenomena may contribute to $\mathcal{L}$ with, for example

\begin{itemize}
\item gain of heat by viscous dissipation of movement 
\item radiation
\item convection   
\end{itemize}


\subsection{Euler's Equation in Conservative Form}

In component form equation \ref{eq:continuity} and \ref{eq:euler} become

\begin{equation*}
  v_{i} \frac{\partial \rho}{\partial t}=-v_{i} \frac{\partial}{\partial x_{j}}\left(\rho v_{j}\right)
\end{equation*}

\begin{equation*}
  \rho \frac{\partial v_{i}}{\partial t}+\left(v_{j} \frac{\partial}{\partial x_{j}}\right) v_{i}=-\frac{\partial p}{\partial x_{i}}+\rho f_{i}
\end{equation*}

Summing the equations and simplifying

\begin{equation*}
  \frac{\partial}{\partial t}\left(\rho v_{i}\right)+\frac{\partial}{\partial x_{j}}\left(\rho v_{i} v_{j}\right)=-\frac{\partial p}{\partial x_{j}}+\rho f_{i}
\end{equation*}

If $f_i = 0$ we get

\begin{equation}
  \label{eq:movement_conservative}
  \frac{\partial}{\partial t}\left(\rho v_{i}\right)+\frac{\partial T_{i j}}{\partial x_{i}}=0
\end{equation}

where $T_{ij} = p \delta_{ij} + p v_i v_j$ is the \textbf{momentum flux} tensor.
In fact, the total momentum of an element with volume $V$ is

\begin{equation*}
  p=\int_{V} \rho v_{i} \mathrm{d} V
\end{equation*}

and it's time derivative

\begin{equation*}
  \frac{\partial}{\partial t} \int_{V} \rho v_{i} \mathrm{d} V=-\int_{V} \frac{\partial T_{i j}}{\partial x_{i}} \mathrm{d} V=-\oint_{S} T_{i j} \mathrm{d} S_{i}
\end{equation*}

\subsection{Energy Equation in Conservative Form}

The energy density is

\begin{equation*}
  \rho \epsilon+\frac{1}{2} \rho v^{2}
\end{equation*}

So one may write

\begin{equation*}
  \frac{\partial}{\partial t}\left(\rho \epsilon+\frac{1}{2} \rho v^{2}\right)=-\frac{\partial}{\partial x_{j}}\left(\epsilon \rho v_{j}\right)+\frac{\partial}{\partial x_{j}}\left(p v_{j}\right)+\frac{\partial}{\partial x_{j}}\left(k \frac{\partial T}{\partial x_{j}}\right)-\frac{1}{2}\left(\delta_{i j} v_{j} v_{i}\right) \frac{\partial}{\partial x_{j}}\left(\rho v_{j}\right)-\rho v_{i}\left(v_{j} \frac{\partial}{\partial x_{j}}\right) v_{i} v_{i}
\end{equation*}

and simplifying

\begin{equation}
  \label{eq:energy_conservative}
  \frac{\partial}{\partial t}\left(\rho \epsilon+\frac{1}{2} \rho v^{2}\right)-\nabla \cdot\left[\rho \vec{v}\left(\frac{1}{2} v^{2}+w\right)-k \nabla T\right]
\end{equation}

where $w = \epsilon + \frac{p}{\rho}$ is the \textbf{enthalpy density}.


\section{Lecture 4 - 26/09/2019}

\subsection{Bernoulli's Principle}

For stationary stream $\left( \pdv{\vec{v}}{t} = 0 \right)$, we may define
current lines that are tangent to $\vec{v}$ at every point. If the stream is
stationary, the current lines trace the path of a given element of fluid. Let
$\vec{f} = \vec{f}_{\text{grav}} = - \nabla \Phi$ where $\Phi$ is the
gravitational potential.

Considering the identity

\begin{equation*}
  \nabla(\vec{A} \cdot \vec{B})=\vec{A} \times(\nabla \times \vec{B}) +  \vec{B} \times(\nabla \times \vec{A})+(\vec{A} \cdot \nabla) \vec{B}+(\vec{B} \cdot \nabla) \vec{A}
\end{equation*}

Taking Euler's equation for a stationary fluid (eq. \ref{eq:euler_ideal_fluid})

\begin{equation*}
  (\vec{v} \cdot \nabla) \vec{v}=-\frac{1}{\rho} \nabla p+\vec{f} 
\end{equation*}

Using the previous identity

\begin{equation*}
  \nabla\left(\frac{v^2}{2}\right)-\vec{v} \times(\nabla \times \vec{v})=-\frac{1}{\rho} \nabla p-\nabla \Phi
\end{equation*}

Integrating along the path of the current

\begin{equation*}
  \int_{l}\left[\nabla\left(\frac{v}{2}\right)-\vec{v} \times(\nabla \times \vec{v})+\frac{1}{\rho} \nabla p+\nabla \Phi\right] \dd{\vec{l}}  =0
\end{equation*}

which gives

\begin{equation}
  \label{eq:bernoulli_general}
  \frac{1}{2} v^{2}+\int \frac{\mathrm{d} p}{\rho}+\Phi=\mathrm{const}
\end{equation}

In the case of an \textbf{incompressible fluid} $\int \frac{\dd{p}}{\rho} =
\frac{p}{\rho}$ we get

\begin{equation}
  \label{eq:bernoulli_incompressible}
  \frac{1}{2} v^{2}+\frac{p}{\rho}+g h=\mathrm{const}
\end{equation}


\end{document}